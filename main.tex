\documentclass{article}
\usepackage[utf8]{inputenc}

\title{Laboratorio01_INTELIGENCIA_NEGOCIOS}
\author{edwartbalcon}
\date{Septiembre 2021}

\usepackage[utf8]{inputenc}
\usepackage[spanish]{babel}
\usepackage{natbib}
\usepackage{graphicx}

\begin{document}

\title{Caratula}

\begin{titlepage}
\begin{center}
\begin{Large}
\textbf{UNIVERSIDAD PRIVADA DE TACNA} \\
\end{Large}
\vspace*{-0.025in}
\begin{figure}[htb]
\begin{center}
\includegraphics[width=6cm]{./images/logo_UPT}
\end{center}
\end{figure}
\vspace*{-0.025in}
\begin{Large}
\textbf{FACULTAD DE INGENIERIA} \\
\end{Large}
\vspace*{0.05in}
\begin{Large}
\textbf{Escuela Profesional de Ingeniería de Sistema} \\
\end{Large}


\vspace*{0.4in}

\vspace*{0.1in}
\begin{Large}
\textbf{Informe de laboratorio 01: Utilizando expresiones de tabla} \\
\end{Large}

\vspace*{0.3in}
\begin{Large}
\textbf{Curso: Inteligencia de negocios} \\
\end{Large}

\vspace*{0.3in}
\begin{Large}
\textbf{DOCENTE: Ing. Patrick Cuadros Quiroga} \\
\end{Large}

\vspace*{0.2in}
\vspace*{0.1in}
\begin{large}

\begin{Large}
\textbf{Alumno: Balcon Coahila, Edwart Juan\hfill	(2013046516) } \\
\end{Large}

\vspace*{0.15in}
\begin{Large}
\textbf{Tacna – Perú} \\
\end{Large}

\vspace*{0.05in}
\begin{Large}
\textbf{2021 } \\
\end{Large}

\end{large}
\end{center}

\end{titlepage}


\newpage

\section{Instalación}

\textbf{1.1. Dependiendo de la elección del producto, descargue el software en la computadora. Después de aceptar
el acuerdo de licencia, puede verificar la instalación haciendo clic en el ícono de Tableau. Si aparece la
siguiente pantalla, está listo para comenzar.}

    \begin{center}
		\includegraphics[width=14cm]{./images/1.1} 
	\end{center}

\newpage

\section{Comenzar}

\textbf{2.1. Espacio de trabajo de Tableau}

    \begin{center}
		\includegraphics[width=14cm]{./images/2.1} 
	\end{center}
	
	
\newpage
\textbf{2.2. Conexión a una fuente de datos}

    \begin{center}
		\includegraphics[width=14cm]{./images/2.2.1(1)} 
	\end{center}
	 \begin{center}
		\includegraphics[width=14cm]{./images/2.2.1(2)} 
	\end{center}
	 \begin{center}
		\includegraphics[width=14cm]{./images/2.2.2} 
	\end{center}
	 \begin{center}
		\includegraphics[width=14cm]{./images/2.2.3 (1)} 
	\end{center}
	 \begin{center}
		\includegraphics[width=14cm]{./images/2.2.3 (2)} 
	\end{center}
	
\newpage
\section{Crear una vista}

\textbf{3.1. Vaya a la hoja de trabajo. Haga clic en la pestaña Sheet 1 en la parte inferior izquierda del
espacio de trabajo del cuadro.}

    \begin{center}
		\includegraphics[width=14cm]{./images/3.1} 
	\end{center}
	
	\newpage
\textbf{3.2. Una vez que esté en la hoja de trabajo, desde Dimensionsdebajo del panel Datos,
arrastre Order Dateal estante Columna}

    \begin{center}
		\includegraphics[width=14cm]{./images/3.2} 
	\end{center}
	\newpage
\textbf{3.3. Del mismo modo, desde la Measurespestaña, arrastre el Salescampo al estante Filas.}

    \begin{center}
		\includegraphics[width=14cm]{./images/3.3} 
	\end{center}
	 \begin{center}
		\includegraphics[width=14cm]{./images/3.4} 
	\end{center}

\newpage


\section{Refinando la vista}

\textbf{4.1. Categoryestá presente en el panel Dimensiones.}

    \begin{center}
		\includegraphics[width=14cm]{./images/4.1} 
	\end{center}
	\begin{center}
		\includegraphics[width=14cm]{./images/4.2} 
	\end{center}
	\begin{center}
		\includegraphics[width=14cm]{./images/4.3} 
	\end{center}
	\begin{center}
		\includegraphics[width=14cm]{./images/4.4} 
	\end{center}
	

\textbf{4.2. La vista por encima de Niza los espectáculos sales de category, por ejemplo, muebles, equipos
de oficina, y la tecnología}

    \begin{center}
		\includegraphics[width=14cm]{./images/4.5} 
	\end{center}
	 \begin{center}
		\includegraphics[width=14cm]{./images/4.6} 
	\end{center}
	 \begin{center}
		\includegraphics[width=14cm]{./images/4.7} 
	\end{center}
	
\newpage
\section{Enfatizando los resultados}

\textbf{5.1. Agregar filtros a la vista}

    \begin{center}
		\includegraphics[width=14cm]{./images/5.1} 
	\end{center}
	
\newpage
\textbf{5.2. Agregar colores a la vista}

    \begin{center}
		\includegraphics[width=14cm]{./images/6.1} 
	\end{center}
	 \begin{center}
		\includegraphics[width=14cm]{./images/6.2} 
	\end{center}

\section{Resultados clave}

\textbf{6.1. En la vista, en la Sub-Category tarjeta de filtro, desactive todas las casillas excepto Bookcases, Tables, y Machines}

    \begin{center}
		\includegraphics[width=14cm]{./images/7.1} 
	\end{center}
\newpage

\textbf{6.2. Seleccione Allen la Sub-Categorytarjeta de filtro para mostrar todas las subcategorías nuevamente.}

    \begin{center}
		\includegraphics[width=14cm]{./images/7.2} 
	\end{center}
\newpage

\textbf{6.3. Desde Dimensiones, arrastre Regional Rowsestante y colóquelo a la izquierda de la pestaña
Suma (Ventas) }

    \begin{center}
		\includegraphics[width=14cm]{./images/7.3} 
	\end{center}
\newpage
\textbf{6.4. . Démosle ahora un nombre a la hoja. En la parte inferior izquierda del espacio de trabajo, haga
doble clic Sheet 1y escriba Sales by Product and Region }

    \begin{center}
		\includegraphics[width=14cm]{./images/7.4} 
	\end{center}
\newpage
\textbf{6.5. Para conservar la vista, Tableau nos permite duplicar nuestra hoja de trabajo para que podamos
continuar en otra hoja desde donde la dejamos.}

    \begin{center}
		\includegraphics[width=14cm]{./images/7.5} 
	\end{center}
\newpage
\textbf{6.6. En su libro de trabajo, haga clic con el botón derecho en la Sales by Product and
Regionhoja y seleccione Duplicatey cambie el nombre de la hoja duplicada a SalesSouth. }

    \begin{center}
		\includegraphics[width=14cm]{./images/7.6} 
	\end{center}
\newpage
\textbf{6.7. En la nueva hoja de trabajo, desde Dimensiones, arrastre Regional Filtersestante para
agregarlo como un filtro en la vista. }

    \begin{center}
		\includegraphics[width=14cm]{./images/7.7} 
	\end{center}
\newpage
\textbf{6.8. En el cuadro de diálogo Región de filtro, desactive todas las casillas de verificación excepto Sur
y luego haga clic en OK. }

    \begin{center}
		\includegraphics[width=14cm]{./images/7.8} 
	\end{center}
\newpage
\textbf{6.9. Por último, no olvide guardar los resultados seleccionando File > Save As. Nombremos
nuestro libro de trabajo como Regional Sales and Profits }

    \begin{center}
		\includegraphics[width=14cm]{./images/7.9} 
	\end{center}
	\begin{center}
		\includegraphics[width=14cm]{./images/7.10} 
	\end{center}
\newpage


\section{Vista de mapa}

\textbf{7.1. Crea una nueva hoja de trabajo}

    \begin{center}
		\includegraphics[width=14cm]{./images/8.1} 
	\end{center}
	\newpage
\textbf{7.2. Agregue State y Country en el panel Datos a Detail en la tarjeta Marcas. Obtenemos la
vista del mapa.}

    \begin{center}
		\includegraphics[width=14cm]{./images/8.2} 
	\end{center}
	\newpage
\textbf{7.3. Arrastre Region a la Filters estantería y luego filtre hacia abajo Southsolo. La vista del
mapa ahora se acerca solo a la región Sur y una marca representa cada estado.
}

    \begin{center}
		\includegraphics[width=14cm]{./images/8.3} 
	\end{center}
	\newpage
\textbf{7.4. Arrastre la Sales medida a la Colorpestaña de la tarjeta Marcas. Obtenemos un mapa relleno
con los colores que muestra el rango de ventas en cada estado}

    \begin{center}
		\includegraphics[width=14cm]{./images/8.4} 
	\end{center}
	\newpage
\textbf{7.5. Podemos cambiar el esquema de color haciendo clic Color en la tarjeta Marcas y
seleccionando Edit Colors. Podemos experimentar con las paletas disponibles.}

    \begin{center}
		\includegraphics[width=14cm]{./images/8.5} 
	\end{center}
	\newpage
\textbf{7.6. Observamos que Florida se está desempeñando mejor en ventas. Si pasamos el cursor sobre
Florida, muestra un total de 89,474 USD en ventas, en comparación con Carolina del Sur, por
ejemplo, que tiene solo 8,482 USD en ventas.}

    \begin{center}
		\includegraphics[width=14cm]{./images/8.6} 
	\end{center}
\newpage
\textbf{7.7. Arrastre Profit hacia Color en la tarjeta Marcas. Ahora vemos que Tennessee, Carolina del
Norte y Florida tienen ganancias negativas, aunque parecía que les estaba yendo bien en
Ventas. Cambiar el nombre de la hoja como Profit Map}

    \begin{center}
		\includegraphics[width=14cm]{./images/8.7} 
	\end{center}
	 \begin{center}
		\includegraphics[width=14cm]{./images/8.8} 
	\end{center}
	 \begin{center}
		\includegraphics[width=14cm]{./images/8.9} 
	\end{center}
	
\newpage	
\section{Entrar en los detalles}

\textbf{8.1. Duplique la hoja de trabajo Mapa de beneficios y asígnele el nombre Gráfico de barras de
beneficios negativos..}

    \begin{center}
		\includegraphics[width=14cm]{./images/9.1} 
	\end{center}
\newpage	
\textbf{8.2. Haga clic Show Meen la hoja de trabajo Gráfico de barras de ganancias negativas.}

    \begin{center}
		\includegraphics[width=14cm]{./images/9.2} 
	\end{center}
\newpage	
\textbf{8.3. Podemos seleccionar más de una barra a la vez simplemente haciendo clic y arrastrando el
cursor sobre ellas. Queremos centrarnos únicamente en los tres estados, es decir, Tennessee,
Carolina del Norte y Florida. Por lo tanto, solo seleccionaremos las barras correspondientes.}

    \begin{center}
		\includegraphics[width=14cm]{./images/9.3} 
	\end{center}
\newpage	
\textbf{8.4. En el panel Datos, arrastre un campo y suéltelo directamente encima de otro campo o haga clic
con el botón derecho en el campo y seleccione.}

    \begin{center}
		\includegraphics[width=14cm]{./images/9.4} 
	\end{center}
\newpage	
\textbf{8.5. Arrastre cualquier campo adicional a la jerarquía. Los campos también se pueden reordenar en
la jerarquía simplemente arrastrándolos a una nueva posición. En la visualización
actual. Crearemos las siguientes jerarquías: Ubicación, Pedido y Producto.}

    \begin{center}
		\includegraphics[width=14cm]{./images/9.5} 
	\end{center}
\newpage	
\textbf{8.6. En el estante de filas, haga clic en el icono con forma de más en el Statecampo para desglosar
el Citynivel}

    \begin{center}
		\includegraphics[width=14cm]{./images/9.6} 
	\end{center}
\newpage	
\textbf{8.7. Eso es una gran cantidad de datos. Podemos usar N-Filter para filtrar y revelar los que tienen
un desempeño más débil}

    \begin{center}
		\includegraphics[width=14cm]{./images/9.7} 
	\end{center}
\newpage	
\textbf{8.8. En el estante Filtros, haga clic con el botón derecho en el conjunto Inclusiones (país, estado) y
seleccione Add to Context. Encontramos que ahora Concord ( Carolina del Norte )
aparece a la vista mientras Miami ( Florida ) han desaparecido. Esto tiene sentido ahora.}

    \begin{center}
		\includegraphics[width=14cm]{./images/9.8} 
	\end{center}
\newpage	
\textbf{8.9. Pero Jacksonville ( Carolina del Norte ) todavía está presente, lo cual es incorrecto. En el
estante Filas, haga clic en el icono con forma de más en la Citypestaña para profundizar en el
nivel de Código postal. Haga clic con el botón derecho en el código postal de Jacksonville, NC,
28540, y luego seleccione Excludepara excluir Jacksonville manualmente .}

    \begin{center}
		\includegraphics[width=14cm]{./images/9.9} 
	\end{center}
\newpage	
\textbf{8.10. Arrastre Código postal del estante Filas. Esta es la vista final.}

    \begin{center}
		\includegraphics[width=14cm]{./images/9.10} 
	\end{center}
	 \begin{center}
		\includegraphics[width=14cm]{./images/9.11} 
	\end{center}
	\begin{center}
		\includegraphics[width=14cm]{./images/9.12} 
	\end{center}
	\begin{center}
		\includegraphics[width=14cm]{./images/9.13} 
	\end{center}
	
\newpage	
\section{Resultados clave}

\textbf{9.1. Arrastre Sub-Categorya las Filas para profundizar más.}

    \begin{center}
		\includegraphics[width=14cm]{./images/10.1} 
	\end{center}
\newpage	
\textbf{9.2. Del mismo modo, arrastre Profithacia Coloren la tarjeta Marcas. Esto nos permite detectar
rápidamente productos con beneficios negativos.}

    \begin{center}
		\includegraphics[width=14cm]{./images/10.2} 
	\end{center}
\newpage
\textbf{9.3. Haga clic derecho en Order Datey seleccione Show Filter. Parece que las máquinas, las
tablas y las carpetas funcionan mal. ¿Entonces, qué debemos hacer? ¿Una solución sería detener
la venta de estos productos en Jacksonville, Concord, Burlington, Knoxville y
Memphis? Verifiquemos si nuestra decisión es correcta.
}

    \begin{center}
		\includegraphics[width=14cm]{./images/10.3} 
	\end{center}
\newpage
\textbf{9.4. Regresemos a la Profit Mappestaña de la hoja creada anteriormente .
}

    \begin{center}
		\includegraphics[width=14cm]{./images/10.4} 
	\end{center}
\newpage
\textbf{9.5. Ahora, haga clic en el Sub-Categorycampo para seleccionar la Show Filteropción}

    \begin{center}
		\includegraphics[width=14cm]{./images/10.5} 
	\end{center}
\newpage
\textbf{9.6. Arrastre Profitdesde abajo Measureshasta la Labeltarjeta Marcas.}

    \begin{center}
		\includegraphics[width=14cm]{./images/10.6} 
	\end{center}
\newpage
\textbf{9.7. Nuevamente, haga clic en Order Datey seleccione Show Filter. Del filtro, eliminemos los
elementos que creemos que contribuyen al beneficio negativo.}

    \begin{center}
		\includegraphics[width=14cm]{./images/10.7} 
	\end{center}
    \begin{center}
		\includegraphics[width=14cm]{./images/10.8} 
	\end{center}
    \begin{center}
		\includegraphics[width=14cm]{./images/10.9} 
	\end{center}
	\begin{center}
		\includegraphics[width=14cm]{./images/10.10} 
	\end{center}
\newpage

\section{Tablero}

\textbf{10.1. Haga clic en el New dashboard botón.}

    \begin{center}
		\includegraphics[width=14cm]{./images/11.1} 
	\end{center}
\newpage	
\textbf{10.2. Arrastra Sales in the South al tablero vacío}

    \begin{center}
		\includegraphics[width=14cm]{./images/11.2} 
	\end{center}
\newpage	

\textbf{10.3.  Arrastre Profit Map al tablero y suéltelo encima de Ventas en la vista Sur. Ambas vistas se
pueden ver a la vez. Para poder presentar los datos de manera que otros puedan entenderlos,
podemos organizar el tablero a nuestro gusto.}

    \begin{center}
		\includegraphics[width=14cm]{./images/11.3} 
	\end{center}
\newpage	

\textbf{10.4. En la Sales Southhoja de trabajo en la vista del tablero, haga clic debajo de Regiony
borre Show Header. Repita el mismo proceso para todos los demás encabezados. Esto ayuda a
enfatizar solo lo que se necesita y oculta la información no tan importante}

    \begin{center}
		\includegraphics[width=14cm]{./images/11.4} 
	\end{center}
\newpage	

\textbf{10.5. En el Profit Map, Ocultar el título también y realizar los mismos pasos para el Sales
Southmapa}

    \begin{center}
		\includegraphics[width=14cm]{./images/11.5} 
	\end{center}
\newpage	

\textbf{10.6. Podemos ver que la Sub-Categorytarjeta de filtro y Year of Order Datese han repetido
en el lado derecho. Eliminemos los extras simplemente tachándolos. Finalmente, haga clic en
el Year of Order Date. Aparece una flecha desplegable y seleccione la opción de Single 
Value (Slider). Ahora deja que la magia se desarrolle. Experimente eligiendo diferentes
años en el control deslizante y las Ventas también variarán en consecuencia.
}

    \begin{center}
		\includegraphics[width=14cm]{./images/11.6} 
	\end{center}
\newpage	

\textbf{10.7.Arrastre el SUM(Profit) filtro a la parte inferior del panel debajo de Ventas en el sur para
obtener una mejor vista.
}

    \begin{center}
		\includegraphics[width=14cm]{./images/11.7} 
	\end{center}
	 \begin{center}
		\includegraphics[width=14cm]{./images/11.8} 
	\end{center}
	 \begin{center}
		\includegraphics[width=14cm]{./images/11.9} 
	\end{center}
	 \begin{center}
		\includegraphics[width=14cm]{./images/11.10} 
	\end{center}
	 \begin{center}
		\includegraphics[width=14cm]{./images/11.11} 
	\end{center}
\newpage	





\section{Añadiendo interactividad}

\textbf{11.1. Comencemos con el Profit Map. Al hacer clic en el mapa, Use as
filter aparece un icono en la parte superior derecha. Haz click en eso. Si seleccionamos
cualquier mapa, las Ventas correspondientes a ese estado se resaltarán en el SalesSouthmapa.}

    \begin{center}
		\includegraphics[width=14cm]{./images/12.1} 
	\end{center}
\newpage	
\textbf{11.2. Para el Year of Order Date, haga clic en la opción desplegable y vaya a Apply to
Worksheets > Selected Worksheets. Se abre un cuadro de diálogo. Seleccione
la Allopción seguida de OK. ¿Qué hace esta opción? Aplica filtros a todas las hojas de trabajo
que tienen la misma fuente de datos.}

    \begin{center}
		\includegraphics[width=14cm]{./images/12.2} 
	\end{center}
\newpage
\textbf{11.3. Explore y experimente. En la visualización a continuación, podemos filtrar el Sales
Southmapa para ver los productos que se venden solo en Carolina del Norte. Luego, podemos
explorar fácilmente las ganancias anuales.}

    \begin{center}
		\includegraphics[width=14cm]{./images/12.3} 
	\end{center}
\newpage
\textbf{11.4. Cambie el nombre del panel a Regional Sales and Profit.
}

    \begin{center}
		\includegraphics[width=14cm]{./images/12.4} 
	\end{center}
	\begin{center}
		\includegraphics[width=14cm]{./images/12.5} 
	\end{center}
	\begin{center}
		\includegraphics[width=14cm]{./images/12.6} 
	\end{center}
\newpage
\end{document}